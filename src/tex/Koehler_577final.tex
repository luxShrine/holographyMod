\documentclass[11pt]{penrose}

\usepackage{amsmath}
% \usepackage{mathsphystools}
% \usepackage{thmstyles}
% \usepackage{graphicx}
% \graphicspath{{images/}}
% ------------------------------------------------------------
%                 Packages
% ------------------------------------------------------------
\usepackage{graphicx}                 % includegraphics
\usepackage{siunitx}                  % \SI{<num>}{<unit>} for µm, nm, …
\usepackage{booktabs}                 % professional tables
\usepackage{subcaption}               % sub‑figures
\usepackage{hyperref}                 % hyperlinks
\usepackage{xcolor}                   % colours for code
\usepackage{listings}                 % source‑code blocks
\usepackage{csquotes}                 % recommended for biblatex

% ------------------------------------------------------------
%                 Document meta‑data
% ------------------------------------------------------------
\title{Deep‑Learning Autofocus for Digital Holography: holographyMod}
\subtitle{Final Project Report}
\author{James Koehler\\jlkoehler@albany.edu}
\date{\today}
\begin{document}

% ------------------------------------------------------------
% Code listing style
% ------------------------------------------------------------
\lstdefinestyle{python}{%
  language=Python,
  basicstyle=\ttfamily\small,
  keywordstyle=\color{blue!70!black},
  commentstyle=\color{gray},
  stringstyle=\color{orange!80!black},
  numberstyle=\tiny\color{gray},
  numbers=left,
  stepnumber=1,
  frame=single,
  breaklines=true,
  tabsize=2
}

% ============================================================
%                      Document
% ============================================================
\begin{document}
\maketitle

\begin{abstract}
  Utilization of a convolutional‑neural‑network (CNN) for the purposes of
  predicting the in‑focus reconstruction depth of digital holograms can improve the
  workflow of research in digital holographic microscopy (DHM). The software
  implementation in my work here is done primarily with Python, and PyTorch 2. This code
  lives in the \texttt{src/holo} package hierarchy. The primary contributions of this
  code are a series of self-maintaining checks on the provided dataset, a configurable
  training script accessed via the CLI package \texttt{Typer}, which provides a rich set
  of diagnostic plots (MAE, residuals, violins, hex‑bins).
\end{abstract}

\tableofcontents
\newpage

% ------------------------------------------------------------
\section{Introduction}
\subsection{Background}
\hspace*{6mm} \newthought{From the time of Aristotle and onwards}, humanity has sought to
capture and understand
what exactly it is we see. With the invention of lenses we had a basis for capturing more
than what the human eye itself was capable of, but we still had enormous issues to
surmount. There was the issue of how exactly to compound the lens as to increase
magnification, how to limit the aberration hitting a lens, and how to effectively
illuminate that which is so small even capturing light from from the specimen can be a
struggle \autoref{microscope_com}.

With the advent of modern computing came the necessity to digitize all forms of
information that was only in the physical world, one important subject was that of images
and those images of microscopy. How DHM as a process works is we record the interference between a
reference and an object beam, yielding an intensity‑only hologram \autoref{fig:laser_setup}. To
recover the complex field one must utilize the output given, alongside the experimental
setup to solve for how the wave that formed such an intensity pattern ought to be. This
is done numerically, by propagating the hologram
to its focus plane. By retrieving this distance and focusing at the ideal distance, we
effectively gain more information that would otherwise never be had. Manual searching for
such a depth is extremely tedious as the process involves the collection of numerous
intensities at various depths, from which only one ought to give the greatest amount of
information, and the difference from it's neighbors would be miniscule
\autoref{Convolutional Neural Network (CNN) vs Vision Transformer (ViT) for Digital Holography}.

Where even complex and robust algorithms fail, there is an opening for neural networks to
attempt to fill in. In the past quarter of this century, we have seen artificial
intelligence go from a mostly philosophical question, to a real application with broad
use cases, such as imaging \autoref{quartz}. A revelation took place with the successful
results from ImageNet classification in 2017, where researchers had successfully employed
a CNN to correctly classify millions of images based on what the subject of the image was
\autoref{ImageNet}. These models have been "increasingly applied in the digital
post-processing of optical signals" effectively serving as both the foundation and
inspiration for this work \autoref{TorchOptics: An open-source Python library for
differentiable Fourier optics simulations}.

The idea that classification of enormous amounts of complicated visual data could be done
by anything other a human working with sophisticated and tailor made computer vision
tools was seldom given real thought. Neural networks have since seen an exponential
amount of use and improvements, leaving the door open for use in tasks such as finding
that desired focus depth.

\subsection{Implementation}
\hspace*{6mm} Therefore I put this tool to work on that exact purpose, feeding a CNN the
labelled holograms on a magnitude that would take years for an individual researcher to
process on their own, is sent to the model over the course of minutes. The information it
gathers and iterates on is all in service of training the model to either classify the
image's depth to regress the optimal propagation distance \$z\$ directly from the raw
hologram. In this way the model is essentially taking guesses and seeing what inferences
and approaches best serve getting that correct answer.

With the training completed, the model can be saved for later use, thus allowing for
training not only on a singular set of data, but various other setups, implementations
etc. This allows for a broad application not only limited to whatever one individual
researcher might have on hand. This solves one of the issues described previously, the
rigidity of traditional methods of autofocus lack broad application. The ability for a
CNN to be adapted to its environment by simply being surrounded by new information to
update it's weights allows for a potential upside to such a method.

This however creates a new set of information to improve, that of the data the CNN is
trained on. It is essential that any baseline set of data be chosen wisely. The neural
network is not like a static computer vision tool that ought to give you the same result
when repeatedly applied to the same sample, the learning nature of an AI makes it
necessary that some randomness exist. Thus, to minimize the "bad" randomness, we can
alter the data upon which it is trained, and can weed out any information that might push
the network down an incorrect path.

% ------------------------------------------------------------

% ------------------------------------------------------------
\section{Methodology}

When training a model there are some high level attributes to tune, such as the
percentage split between testing and validation. I go with a default of $80-20$, with the
majority going toward training, which is a standard split \autoref{Convolutional Neural
Network (CNN) vs Vision Transformer (ViT) for Digital Holography}.

\subsection{Network architecture}
The backbone is I implement is primarily \textbf{EfficientNet‑B4} (\cite{tan2019efficientnet})
pre‑trained on ImageNet; the classification head is replaced by a
single linear neuron for regression:
\begin{lstlisting}[style=python,caption={Head replacement in \texttt{epoch\_helper.get\_model()}}]
selected_model = models.efficientnet_b4(weights=...)  # 179M params
in_feats = selected_model.classifier[1].in_features
selected_model.classifier[1] = nn.Linear(in_feats, 1)
\end{lstlisting}
What this means practically is that the model's classification leads to a single
prediction rather than trying to classify the image into a set bin of $z_{depth}$.
The main reason why I went with this model is due to my code initially being focused
solely on classification and this being a relatively small model means it can easily
applied. However, due to limitations with respect to the availability of well labeled and
large datasets, the span of my data was quite low. Meaning that my code was effectively
limited in the number of bins to classify into, thus leading to my refactoring of the
code to primarily utilize a regression approach.

\subsection{Training procedure}
To implement such an approach my code utilizes a few core torch functionalities when
training the model. First my code loads in data from a selected csv file which contains
the mappings to each image in the dataset along with its relevant attributes such as the
wavelength of light used in imaging and the depth of $z$ for the hologram. My code has a
number of checks against potentially corrupt or incomplete data.

In the \texttt{HologramFocusDataset} class I check if any of the images in the dataset do
not exist, or if the do exist that they are properly encoded. This serves two functions,
where it not only prevents corrupt or missing data from staying stored in any sub class
and being passed on to be trained on, but it also allows for the partial training or
evaluation of the model on a dataset. This second point is not typically what is
considered when performing this kind of error handling, however in the case of imaging
specifically there is a desire to potentially have a limited sample set. Hologram image
files used in microscopy are typically extremely large as to retain the maximum amount of
physical meaning before reconstruction, this would mean that processing a complete
dataset could be very computationally intensive for even a small amount of training.
Thus, partial training allows for the user to get a cursory understanding of how
different model backbones might perform with their own data, and how many system
resources it might require to include to process their whole dataset.

From this dataset class I have two sub classes, one for processing particularities with
the HQ-DLHM-OD-Databse dataset, and another for actually applying the transformation of
the data to, thus preserving the original class and its information for use at any later
point. This approach allows for the easy addition of another class for whatever
particular date the user might have, and simply pointing the parent class to handle
anything that ought to be universal for any dataset.

For the actual training of the model, I utilize another class, a lightweight dataclass
\texttt{AutoConfig}, for everything not explicitly linked to the datasets, this helps
keep the code more segmented and thus allows for easily accessing the information needed
at any part of the training, without having to invoke the dataset class itself which
contains much more information than needed.

\subsubsection{Regression}
To perform regression training their are some key differences from classification to
address, the main one being that of bins. Since the model expects to be given a bin to
place the image depth into, there needs to be some bin that exists. Thus, to circumvent
this in a way that allows for a continuous prediction we simply reduce the number of bins
to one.

To steer the model in the correct direction during its training, we utilize the
criterion Smooth L1 loss with the optimizer AdamW ($\eta=\SI{5e-5}{}$). The loss measurer
essentially gives feedback to the model telling how well it is doing relative to each
epoch. While the optimizer AdamW is a gradient descent algorithm that adjusts the
learning rate to better fit the stream of information. It does so by taking the mean and
variance, and then applying a bias correction to these values, before updating the
weights utilizing this new estimate
\autoref{https://www.geeksforgeeks.org/adam-optimizer/}. This is used in conjunction with
a scheduler, \texttt{ReduceLROnPlateau}, which lowers the learning rate by a factor of 10
if validation MAE stalls for five epochs. These three form the guardrails that guide the
regression model to a better result. All experiments use the seed 42 to ensure
reproducibility, although this function can be removed if this is not desired.

\subsubsection{Classification}
My code is still largely focused on the regression approach as mentioned above, but my
code does allow for classification training to be used simply by passing the \texttt{-c}
flag to the training function. From here the training is switched onto a different track,
still running through the same functions as to make use of their largely shared basis for
training. Of note is that the optimizer and scheduler remain the same, while the
criterion is replaced with \texttt{CrossEntropyLoss}, a measurer of loss better suited to
discrete classification.

Further, bins are now important for classification, therefore in the dataset class I go
through a rigorous test to ensure that the number of bins, width of bins, and quantity in
each bin is satisfactory for classification. My code does not raise an error for the most
part during this, however it does log errors to the console and \texttt{debug.log} file
as to inform the user of the corrections being made or issues that must be ignored.

This process of binning and properly assessing the bins quality, has the potentially for
being computationally more efficient than regression, however none of the above setup is
necessary for the regression setup, which creates a double edged sword. I do not have to
ensure a robust dataset exists which provides a solid foundation for training exists,
instead, I leave it up to the user to decide whether their data is representative enough
to be trained on.

\subsection{Implementation notes}
\begin{itemize}
  \item CLI entry‑point: \texttt{scripts/main.py train}
  \item Progress bars: Rich (Figure\ref{fig:rich})
  \item Unit safety: all physical quantities are stored in \emph{metres}
    (see \texttt{dataset.HologramFocusDataset}).
\end{itemize}
There are some peculiarities of the database I am I using to produce the metrics of how
my my code works. The HQ-DLHM-OD-Databse is one of the few easily accessible, large, and
labeled hologram datasets I could find \autoref{https://osf.io/syht9/}. The dataset was
captured using an 8-bit monochromatic camera, MN34230 sensor, with a 3.8 um pixel size,
with a resolution that is not a square. Thus, I implement checks to ensure the image is
cropped to the largest possible square.

Aside from my data used, I also implemented a progress bar for use during epoch
iterations, as the process can take a long time to complete, and seeing how the model
updates in real time can be important to ensure the model isn't going off in an obviously
wrong direction. It serves this functionality well, as it nicely reads out the current
assigned learning rate, progress on the current batch and epoch, as well as the
validation and training loss. Further from the same \texttt{Rich} package I utilize it as
a logger to display a stopwatch metric of sorts for the validation process, which simply
reads out the time taken for each validation loop. Thus, the user has direct access to
all of the relevant information during training and can make adjustments without having
to wait for a complete training cycle to continue.

To demo the code as a complete package, going dataset to result, I include a small
function to actually reconstruct images provided
and compare how the model's predictions line up with the true or "ground truth" values of
the actual holograms. This functionality is not demoed in this paper however, due to the
lack of well labeled and accessible hologram data that includes both phase and depth measurements.

\begin{figure}[ht]
  \centering
  \includegraphics[width=0.88\textwidth]{figures/rich_bar.png}
  \caption{A sample of how the code's progress bar and indicators look, coded using the
  \texttt{Rich} module.}
  \label{fig:rich}
\end{figure}

% ------------------------------------------------------------
\section{Evaluation metrics}
\begin{description}
  \item[MAE]  Mean average error $|z\_{\text{pred}}-z\_{\text{true}}|$.
  \item[Residual plots]  Signed error vs. ground truth (Fig.\ref{fig:res_test} & \ref{fig:res_val}).
  \item[Violin plots]   Error distribution per depth slice(Fig. \ref{fig:vio_test} &
    \ref{fig:vio_val}) .
  \item[Hex‑bin]        2‑D density of predictions vs. truth.
\end{description}
Due to the lack of robust accuracy evaluations that can be preformed here due to the
limitation on data, a large part of my code's error analysis concerns the self
consistency of error, in that I still recover meaningful metrics concerning the precision
of the model.

To do so I use a wide variety of metrics, the simplest being the mean average error,
which in some cases I reduce further to the normalized root mean squared error (NRMSE) as
to ensure their isn't a massive error between the prediction and the expected. This is
best shown via residual plots referenced above, that show the visual difference from two
distributions. This allows for the user to pick out any particular region that might show
a blindspot of the data or how the model predicts.

To showcase how error varies with prediction range, I utilize violin plots, named for the
shape the curves create, filling in a region of peaks and valleys not unlike a violin.
The area that is plotted showcases if their is variation across different regions, which
could be another indicator of insufficient data or something else. Such a plot can inform
the user about whether the data can be used for inferencing purposes, as you might have a
relatively large amount of error at the extremes of your prediction values, which ought
to soften any conclusions beyond what has already been analyzed.

For an absolute measure of quality, one common and relatively simple metric is the peak
signal to noise ratio
\ref{https://www.researchgate.net/publication/303517356_Comparison_of_image_quality_assessment_PSNR_HVS_SSIM_UIQI}.
This essentially measures the total magnitude of output pixels, compared to what the
maximum could have been. Which gives you a measure of how much information you recovered
relative to the total information available to you.

% ------------------------------------------------------------
\section{Results}\label{sec:results}

My code can complete $30$ epochs after two hours on my system, using the \texttt{efficientnet\_b4}
model with a learning rate of $1e-4$, crop size of $256$, batch size of $16$, I was able
to see a massive improvement from epoch to epoch as the model consistently improved as
more time elapsed, starting at a training loss of $0.00151519$, evaluation loss of
$0.0035268$ and mean average error of $0.0000000109$, to train loss $0.00060473$, val
loss $0.00016834$, and val Mae of $0.0000000076$ after one epoch. The results shown here
in the plots are from the model after completing the $30$ epochs however.

The residual plots don't describe any particular meaningful error, it seems as if the
training prefers to undervalue the predicted values. Further, I have left in the
classification hex binning plots \ref{fig:hex_test} and \ref{fig:hex_val}, simply because
they do display the distribution of the predictions, which follow a normal curve, showing
that we have successfully prevented any systematic bias from occurring.

To see what a sample of classification could give, I do provide the model's
classification training  plot from an earlier iteration of the code, which should serve
as a proof of concept concerning the potentiality of utilizing this alongside a robust
dataset \ref{fig:classificaton}.

\begin{figure}[ht]
  \centering
  \includegraphics[width=0.48\textwidth]{figures/residual_vs_true_train.png}
  \caption{Residual error as a function of true depth.}
  \label{fig:res_test}
\end{figure}

\begin{figure}[ht]
  \centering
  \includegraphics[width=0.48\textwidth]{figures/residual_vs_true_val.png}
  \caption{Residual error as a function of true depth.}
  \label{fig:res_val}
\end{figure}

\begin{figure}[ht]
  \centering
  \includegraphics[width=0.48\textwidth]{figures/error_violin_train.png}
  \caption{Residual error as a function of true depth.}
  \label{fig:vio_test}
\end{figure}

\begin{figure}[ht]
  \centering
  \includegraphics[width=0.48\textwidth]{figures/error_violin_val.png}
  \caption{Residual error as a function of true depth.}
  \label{fig:vio_val}
\end{figure}

\begin{figure}[ht]
  \centering
  \includegraphics[width=0.48\textwidth]{figures/focus_depth_actual_vs_pred_50ep_sig1_heat.png}
  \caption{Residual error as a function of true depth.}
  \label{fig:classificaton}
\end{figure}

% ------------------------------------------------------------
\section{Conclusion and future work}

For the limited training done, the network converged to the global mean when trained on the
range of the original data (see Figure\ref{fig:hex}).  Enlarging
the range to \SI{0}{\micro\metre}–\SI{40}{\micro\metre} or introducing
physics‑based augmentations to the dataset is expected to alleviate this collapse onto
one value. This is assumed to be the case due to the robust metrics that have been output
by the model showing that the internals of the system produce high precision results.

We demonstrated an end‑to‑end pipeline for learning autofocus depths
from raw holograms.  While initial results highlight the challenges of
regressing over such a small span, the framework is ready for larger or
synthetic datasets and provides the framework for formulations combining regression
and classification.

The framework of my program is setup such that it can easily be expanded upon with the
framework, dataset, and even goal agnostic approach I have taken here. With a proper goal
in mind, it is easy to see how my framework can be easily deployed to help streamline the
reconstruction process.

To truly put the model to the test, it could be implemented in a live environment where
information is being sent directly from a lab DHM setup to a program making predictions
and updating its weights in real time as imaging is being done, then this could become a
two way street, where the experiment can be improved alongside the autofocus model, in a
symbiotic way.

% ------------------------------------------------------------

% ------------------------------------------------------------
\appendix
\section{Key training command}
\begin{lstlisting}[style=python]
uv run --scripts main.py train&#x20;
\--backbone efficientnet\_b4 --crop 224 --batch 16&#x20;
\--ep 50 --lr 1e-4 --device cuda
\end{lstlisting}

\section{Additional figures}
% add extra plots if necessary

\begin{figure}[ht]
  \centering
  \includegraphics[width=0.48\textwidth]{figures/hexbin_train.png}
  \caption{Residual error as a function of true depth.}
  \label{fig:hex_test}
\end{figure}

\begin{figure}[ht]
  \centering
  \includegraphics[width=0.48\textwidth]{figures/hexbin_val.png}
  \caption{Residual error as a function of true depth.}
  \label{fig:hex_val}
\end{figure}

\end{document}

\begin{center}
  \vspace*{0.5em}
  \rule{0.8\textwidth}{0.8pt}
\end{center}

\nocite{*}
{\small \bibliography{references}}

\end{document}
